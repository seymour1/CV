\documentclass[11pt]{article}
\usepackage{hyperref}
\pagestyle{headings}

\newcommand{\tab}{\hspace*{3em}}
\newcommand{\atab}{\hangindent=3em \hangafter=0}
\newcommand{\btab}{\hangindent=6em \hangafter=0}
\newcommand{\ctab}{\hangindent=9em \hangafter=0}
\newcommand{\untab}{\hspace*{-3em}}

\newenvironment{my_list}
{\begin{itemize}
    \vspace{-4mm}
    \setlength{\itemsep}{0pt}
    \setlength{\itemindent}{6em}
    \setlength{\parskip}{0pt}
    \setlength{\parsep}{0pt}}
{\end{itemize}}

\newenvironment{publications}
{\begin{itemize}
    \vspace{-4mm}
    \setlength{\itemsep}{1pt}
    \setlength{\itemindent}{0em}
    \setlength{\parskip}{0pt}
    \setlength{\parsep}{0pt}}
{\end{itemize}}

%% margins
\oddsidemargin  -0.4in
\textwidth      7in
\topmargin  -0.5in
\headheight 0.0in
\headsep    0.0in
\textheight 11.0in
\pagenumbering{gobble}
\begin{document}

\centering
\normalsize
\medskip
\begin{tabular*}{\textwidth}{c @{\extracolsep{\fill}} c c }
& \LARGE John J. Seymour, III\\
seymour1@umbc.edu & & sites.google.com/site/jjseymour3\\
https://github.com/seymour1/ & & 410-350-4877
\end{tabular*}
\medskip
\hline

\medskip
\begin{tabular}{ l l }
\textsc{Research Interests:} & Machine Learning, Malware Analysis\\
\end{tabular}

\raggedright

\medskip
\textsc{Education}\\
\atab \textbf{University of Maryland, Baltimore County (UMBC)}\\
\btab Ph.D. in Computer Science (Expected Fall 2017)\\
\ctab Graduate GPA: 4.0/4.0\\
\btab M.S. in Computer Science (Fall 2014)\\
\ctab Thesis Title: Quantum Classification of Malware\\
\btab B.S./B.S./B.A. \emph{cum laude} in Computer Science/Mathematics/Philosophy (Fall 2011)\\
\ctab Undergraduate GPA: 3.70/4.00, Certificate of General Honors\\

\medskip
\textsc{Work Experience}

\medskip
\atab \textbf{ZeroFOX}\\
\btab \textit{Senior Data Scientist}, September 2016 - Present\\
\begin{my_list}
\item Piloted ZeroFOX FoxThreats program for threat hunting on social media.
\item Conveyed complex machine learning concepts to press and conference attendees.
\item Interviewed and mentored new hires.
\end{my_list}
\btab \textit{Data Scientist}, September 2015 - September 2016\\
\begin{my_list}
\item Led development of SNAP\_R, a machine learning based pen-testing tool to\\ \tab \tab automate generation of individually tailored phishing messages on Twitter.
\item Developed solution to detect money-flipping scam posts on Instagram.
\item Created high-quality datasets and classifiers for product offerings.
\end{my_list}

\medskip
\atab \textbf{CyberPoint International, LLC}\\
\btab \textit{Student Intern}, Summer 2015\\
\begin{my_list}
\item Researched bleeding-edge malware classification techniques and integrated them\\ \tab \tab into the CyberPoint Machine Learning Model Training Pipeline.
\item Created algorithm and data structure vulnerabilities for DARPA STAC competition.
\end{my_list}

\medskip
\atab \textbf{University of Maryland, Baltimore County}\\
\btab \textit{Graduate Research Assistant, UMBC DREAM Lab}, January 2014 - May 2015\\
\begin{my_list}
\item Scraped urlquery.net for links to websites redirecting to exploit kits.
\item Intercepted and recorded traffic to exploit kit landing pages for use in data analysis.
\end{my_list}

% August 2012 - December 2012: Automata/OOP
% January 2013 - August 2013: Cryptography
% August 2013 - December 2013: Computer/Network Security
\btab \textit{Graduate Teaching Assistant}, August 2012 - December 2013\\
\begin{my_list}
\item Teaching Assistant for Network Security, Computer Security, Cryptography,\\ \tab \tab Automata Theory, and Introduction to Object-Oriented Programming.
\item Taught Metasploit, Kali Linux, and infosec theory to undergraduate students.
\end{my_list}

%Jan 2012 - June 2012: Finin (Voting project)
%Summer 2012, Winter 2012, Summer 2013: SecuEmp
\btab \textit{Graduate Research Assistant, UMBC Cyber Defense Lab}, January 2012 - August 2013\\
\begin{my_list}
\item Designed, deployed, and maintained LAMP stack to host government-funded\\ \tab \tab gamification initiative for teaching high school students basic concepts in infosec.
\end{my_list}

\medskip
\atab \textbf{Army Research Lab}\\
\btab \textit{Student Intern}, Summer 2014\\
\begin{my_list}
\item Transformed satisfiability problems into problems the D-Wave SR10V6 could solve.
\item Demonstrated and reduced bias in D-Wave chips using statistical techniques.
\end{my_list}

\medskip
\atab \textbf{Pyxis Engineering/Applied Signals Technology}\\
\btab \textit{Associate Engineer}, June 2009 - January 2010\\
\begin{my_list}
\item Designed and deployed a Training Request Management System.
\end{my_list}

%------------------------------------------PAGE 2: Projects, Publications, Activities, Honors, etc------------------------------------------------------
\newpage
\centering
\normalsize
\medskip
\begin{tabular*}{\textwidth}{c @{\extracolsep{\fill}} c c }
& \LARGE John J. Seymour, III\\
seymour1@umbc.edu & & sites.google.com/site/jjseymour3\\
https://github.com/seymour1/ & & 410-350-4877
\end{tabular*}
\medskip
\hline
\medskip
\begin{tabular}{ l l }
\textsc{Research Interests:} & Machine Learning, Malware Analysis\\
\end{tabular}

\raggedright

\medskip
\textsc{Publications}\\
\begin{publications}
\item John Seymour, ``Hacking a Corporate Social Media Page'', ZeroFOX white paper, December 2016.
\item John Seymour and Charles Nicholas, ``How to build a malware classifier [that doesn’t suck on real-world data]'', SecTor, October 2016. Also presented at Data Science MD (October 2016).
\item Philip Tully, John Seymour, and Spencer Wolfe, ``Post Grams Not Scams: Detecting Money Flipping Scams on Instagram Using Machine Learning'', ZeroFOX white paper, September 2016. Covered by BBC News/Technology, NBC News, and The Verge, among others.
\item John Seymour and Philip Tully, ``Weaponizing Data Science for Social Engineering: Automated E2E Spear Phishing on Twitter'', Black Hat USA, August 2016. Also presented at DEF CON 24 (August 2016), CyberGamut (December 2016), BrightTalk (December 2016), Baltimore Python (July 2016).  Covered by the Atlantic, MIT Technology Review, and Forbes, among others.
\item John Seymour and Charles Nicholas, ``Labeling the VirusShare Corpus: Lessons Learned'', BSidesLV, August 2016. Also presented at SPARSA South (July 2016), Baltimore Python (June 2016).
\item Alan T. Sherman, John Seymour, Akshayraj Kore and William Newton, ``Chaum's protocol for detecting man-in-the-middle: Explanation, demonstration, and timing studies for a text-messaging scenario'', Cryptologia, May 2016.
\item John Seymour and Charles Nicholas, ``An Introduction to Malware Classification'', BSidesCharm, April 2016.
\item John Seymour and Charles Nicholas, ``Quantum'' Classification of Malware, DEF CON 23, August 2015. Also presented at CyberPoint International (July 2015), UMBC DREAM LAB (July 2015).
\item John Seymour, ``Overgeneralization in Feature Set Selection for Classification of Malware'', CSEE Technical Report TR-CS-14-06, August 2014. Also presented at Malware Technical Exchange Meeting (June 2015, poster session), CyberPoint International (July 2015).
%(Presentation) SecurityEmpire: Development and Evaluation of a Digital Game to Promote Cybersecurity Education, USENIX 3GSE, August 2014. Presented by Dr. Alan Sherman.
\item John Seymour, ``Skew Removal from SAT Instances using the D-Wave Quantum Annealer'', Army Research Laboratory, July 31.
\item Charles Nicholas, Robert Brandon, Joshua Domangue, Andrew Hallemeyer, Peter Olsen, Alison Pfannenstein and John Seymour, ``The Exploit Kit Club'', Malware Technical Exchange Meeting, July 22-24, 2014, Albuquerque, NM.  (poster session)
\item Charles Nicholas, Robert Brandon, Andrew Coates, Andrew Hallemeyer, Brian Hillsley, Victoria Lentz, Edward Mukasey, Peter Olsen, Alison Pfannenstein and John Seymour, ``Tracking Exploit Kit Activity Through Landing Page Analysis'', UMBC Computer Science and Electrical Engineering Department Technical Report 14-005, May 7, 2014.
\item John Seymour, Joseph Tuzo, and Marie desJardins,``Ant Colony Optimization in a Changing Environment,'' Working Notes of the AAAI Fall Symposium on Complex Adaptive Systems, November 2011.
\item Joseph Tuzo, John Seymour and Marie desJardins,  ``Using a Cellular Automaton Simulation to Determine an Optimal Lane Changing Strategy on a Multi-lane Highway,'' Working Notes of the AAAI Fall Symposium on Complex Adaptive Systems, November 2011.
\end{publications}

\medskip
\textsc{Affiliations}\\
\medskip
\begin{tabular}{l l}
Admin: Baltimore Python & Member: MLSecProject\\
Member: MD Data Science & Member: SPARSA South\\
Judge: Code SLAM &  Technical Consultant: UMBC Ethics Bowl Team\\
Volunteer: CMPC Cold Weather Shelter & Member: United States Fencing Association\\
UMBC Honors College Alumn & Eagle Scout\\
\end{tabular}
\end{document}